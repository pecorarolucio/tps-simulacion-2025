
\documentclass[11pt,a4paper]{article}
\usepackage[spanish]{babel}
\usepackage[utf8]{inputenc}
\usepackage[T1]{fontenc}
\usepackage{amsmath, amssymb}
\usepackage{graphicx}
\usepackage{geometry}
\usepackage{hyperref}
\usepackage{float}
\usepackage{listings}
\usepackage{caption}
\usepackage{booktabs}
\usepackage{siunitx}

\geometry{margin=2.5cm}

\lstset{
    language=Python,
    basicstyle=\ttfamily\small,
    keywordstyle=\color{blue},
    commentstyle=\color{gray},
    stringstyle=\color{red},
    showstringspaces=false,
    columns=fullflexible,
    breaklines=true,
    frame=single,
    captionpos=b
}

\title{TP 2.2 -- Generadores de Números Pseudoaleatorios de Distintas Distribuciones de Probabilidad}
\author{Renzo \\ Universidad Tecnológica Nacional -- FRRO}
\date{\today}

\begin{document}

\maketitle

\begin{abstract}
El presente informe describe el desarrollo e implementación en \textbf{Python~3.x} de generadores de números
pseudoaleatorios para varias distribuciones de probabilidad, tanto continuas como discretas, de acuerdo con las
directrices del Trabajo Práctico~2.2.
Se incluyen las bases teóricas, la deducción de las funciones acumuladas e inversas (cuando corresponde),
los algoritmos de generación, las pruebas de bondad de ajuste y comentarios sobre los resultados obtenidos.
Todas las \emph{gráficas de histograma} han sido omitidas en este documento y deberán ser insertadas en los lugares
indicados por el lector, utilizando los archivos de imagen generados por el script \texttt{pruebas.py}.
\end{abstract}

\section{Introducción}
La generación de números pseudoaleatorios es un pilar fundamental en la simulación por computadora.
Si bien el generador congruencial mixto implementado en el TP~2.1 produce valores con distribución uniforme
continua en $[0,1]$, la \emph{mayoría} de los fenómenos que se desean modelar siguen otras distribuciones.
El objetivo de este trabajo es, entonces, transformar la secuencia uniforme en muestras de
las distribuciones mostradas en la Tabla~\ref{tab:listado}.
Se parte de los métodos clásicos recopilados por \cite{naylor1982} y se
desarrolla código en Python que posibilita su empleo en estudios posteriores de la asignatura.

\section{Metodología general}
Para cada distribución se realiza el siguiente procedimiento:
\begin{enumerate}
    \item Investigación teórica: definición, parámetros, función de densidad (o masa) de probabilidad, 
    función acumulada (CDF) e inversa (ICDF) cuando existe forma cerrada.
    \item Selección del \emph{método de generación}: transformación inversa, aceptación--rechazo, 
    composición o producto de variables, según convenga.
    \item Implementación del algoritmo en Python.  
    Todas las rutinas se encuentran en el archivo \texttt{pruebas.py},
    cuya inclusión completa se realiza en el Apéndice~\ref{app:codigo}.
    \item Evaluación estadística mediante la prueba de $\chi^{2}$, con un nivel de significancia
    $\alpha = 0{,}05$.
    \item Visualización del histograma de los valores generados y comparación con la densidad teórica.
\end{enumerate}

\section{Distribuciones estudiadas}
La Tabla~\ref{tab:listado} resume las tareas obligatorias según el enunciado.

\begin{table}[H]
    \centering
    \begin{tabular}{lccccc}
        \toprule
        Distribución & Tipo & T. Inversa & M. Rechazo & Código & Testeo \\
        \midrule
        Uniforme & continua & \checkmark & \checkmark & \checkmark & \checkmark \\
        Exponencial & continua & \checkmark & \checkmark & \checkmark & \checkmark \\
        Gamma & continua & -- & \checkmark & \checkmark & -- \\
        Normal & continua & \checkmark & \checkmark & \checkmark & \checkmark \\
        Pascal & discreta & -- & \checkmark & \checkmark & -- \\
        Binomial & discreta & -- & \checkmark & \checkmark & \checkmark \\
        Hipergeométrica & discreta & -- & \checkmark & \checkmark & -- \\
        Poisson & discreta & -- & \checkmark & \checkmark & \checkmark \\
        Empírica discreta & discreta & -- & \checkmark & \checkmark & \checkmark \\
        \bottomrule
    \end{tabular}
    \caption{Listado de distribuciones y tareas realizadas.}
    \label{tab:listado}
\end{table}

\subsection{Distribución Uniforme continua $U(a,b)$}
\paragraph{Definición}
La densidad es
\[
f(x) = \frac{1}{b-a}, \qquad a\le x \le b .
\]
La CDF es $F(x)=\frac{x-a}{b-a}$ y la inversa se obtiene despejando
$F^{-1}(u)= a + (b-a)u$.

\paragraph{Algoritmo}
Sea $R\sim U(0,1)$ del generador base. Entonces $X = a + (b-a)R$ 
sigue $U(a,b)$.

\paragraph{Implementación}
\begin{lstlisting}[caption={Generador de la distribución uniforme},label={lst:uni}]
def distr_uniforme(a, b, size):
    x = []
    for _ in range(size):
        x.append(a + (b - a) * random.random())
    return x
\end{lstlisting}

\paragraph{Test de bondad de ajuste}
Se usa la prueba $\chi^{2}$ con $k=10$ sub–intervalos.
Los resultados se resumen en la Tabla~\ref{tab:chi2}.

\paragraph{Gráfica}\\
\textbf{Insertar aquí la figura del histograma de la distribución uniforme.}
\begin{figure}[H]
    \centering
    %\includegraphics[width=0.7\textwidth]{figs/uniforme_hist.png}
    \caption{Histograma de muestras $U(5,15)$.}
    \label{fig:uni_hist}
\end{figure}

% ------------------------------------------------------------------------
\subsection{Distribución Exponencial $\operatorname{Exp}(\lambda)$}
\paragraph{Definición}
\[
f(x)=\lambda e^{-\lambda x}, \qquad x\ge 0,\; \lambda>0 .
\]
CDF $F(x)=1-e^{-\lambda x}$; ICDF $F^{-1}(u)=-\frac{1}{\lambda}\ln(1-u)$.

\paragraph{Algoritmo}
Transformación inversa: $X = -\frac{1}{\lambda}\ln R$.

\paragraph{Implementación}
\begin{lstlisting}[caption={Generador exponencial},label={lst:exp}]
def distr_exp(lmbda, size):
    return [-lmbda * math.log(random.random()) for _ in range(size)]
\end{lstlisting}

\paragraph{Test de bondad de ajuste}
Ver Tabla~\ref{tab:chi2}.

\paragraph{Gráfica}\\
\textbf{Insertar aquí el histograma de la distribución exponencial.}
\begin{figure}[H]
    \centering
    %\includegraphics[width=0.7\textwidth]{figs/exp_hist.png}
    \caption{Histograma de muestras exponenciales ($\lambda=0.2$).}
\end{figure}

% ------------------------------------------------------------------------
\subsection{Distribución Gamma $\Gamma(k,\beta)$}
\paragraph{Definición}
Para $k\in\mathbb{N}$,
\[
f(x)=\frac{\beta^{k}}{\Gamma(k)}x^{k-1}e^{-\beta x},\qquad x\ge 0 .
\]

\paragraph{Algoritmo}
Producto de $k$ variables $U(0,1)$ (método de Naylor):
\[
X=-\frac{1}{\beta}\ln\left(\prod_{i=1}^{k}R_i\right).
\]

\paragraph{Implementación}
\lstinputlisting[caption={Generador gamma (fragmento)},
                firstline=46,lastline=60]{pruebas.py}

\paragraph{Gráfica}\\
\textbf{Insertar aquí el histograma de la distribución Gamma.}

% ------------------------------------------------------------------------
\subsection{Distribución Normal $\mathcal{N}(\mu,\sigma^{2})$}
\paragraph{Definición}
\[
f(x)=\frac{1}{\sqrt{2\pi\sigma^{2}}}
      \exp\!\left(-\frac{(x-\mu)^{2}}{2\sigma^{2}}\right).
\]

\paragraph{Algoritmo}
Método de los 12~uniformes (aproximación de Irwin–Hall):
\[
X = \mu + \sigma\left(\sum_{i=1}^{12}R_i - 6\right).
\]

\paragraph{Implementación}
\lstinputlisting[caption={Generador normal (fragmento)},
                firstline=73,lastline=84]{pruebas.py}

\paragraph{Test de bondad de ajuste}
Ver Tabla~\ref{tab:chi2}.

\paragraph{Gráfica}\\
\textbf{Insertar aquí el histograma de la distribución normal.}

% ------------------------------------------------------------------------
\subsection{Distribución Binomial $B(n,p)$}
\paragraph{Definición}
\[
P(X=k)=\binom{n}{k}p^{k}(1-p)^{n-k}, \qquad k=0,\dots,n .
\]

\paragraph{Algoritmo}
Conteo de éxitos en $n$ ensayos de Bernoulli.

\paragraph{Implementación}
\lstinputlisting[caption={Generador binomial (fragmento)},
                firstline=98,lastline=110]{pruebas.py}

\paragraph{Test de bondad de ajuste}
Ver Tabla~\ref{tab:chi2}.

\paragraph{Gráfica}\\
\textbf{Insertar aquí el histograma de la distribución binomial.}

% ------------------------------------------------------------------------
\subsection{Distribución de Poisson $P(\lambda)$}
\paragraph{Definición}
\[
P(X=k)=\frac{\lambda^{k}e^{-\lambda}}{k!},\qquad k=0,1,\dots
\]

\paragraph{Algoritmo}
Método de cumulativos (Knuth).

\paragraph{Implementación}
\lstinputlisting[caption={Generador Poisson (fragmento)},
                firstline=140,lastline=155]{pruebas.py}

\paragraph{Test de bondad de ajuste}
Ver Tabla~\ref{tab:chi2}.

\paragraph{Gráfica}\\
\textbf{Insertar aquí el histograma de la distribución Poisson.}

% ------------------------------------------------------------------------
\subsection{Distribución Pascal (binomial negativa)}
\paragraph{Definición}
Se modela el número de fracasos antes de $k$ éxitos.

\paragraph{Implementación}
\lstinputlisting[caption={Generador Pascal (fragmento)},
                firstline=86,lastline=96]{pruebas.py}

\paragraph{Gráfica}\\
\textbf{Insertar aquí el histograma de la distribución Pascal.}

% ------------------------------------------------------------------------
\subsection{Distribución Hipergeométrica}
\paragraph{Definición}
$X$ cuenta éxitos en una muestra sin reposición.

\paragraph{Implementación}
\lstinputlisting[caption={Generador hipergeométrica (fragmento)},
                firstline=112,lastline=136]{pruebas.py}

\paragraph{Gráfica}\\
\textbf{Insertar aquí el histograma de la distribución hipergeométrica.}

% ------------------------------------------------------------------------
\subsection{Distribución empírica discreta}
\paragraph{Definición}
Se parte de un conjunto finito $\{x_i\}$ con probabilidades asociadas $p_i$.

\paragraph{Implementación}
\lstinputlisting[caption={Generador empírico discreto (fragmento)},
                firstline=157,lastline=176]{pruebas.py}

\paragraph{Gráfica}\\
\textbf{Insertar aquí el histograma de la distribución empírica.}

% ------------------------------------------------------------------------
\section{Resultados de las pruebas $\chi^{2}$}
Ejecutando la función \texttt{inicio()} se imprimen en consola los
estadísticos de contraste.
La Tabla~\ref{tab:chi2} recoge los valores observados
(\emph{completar luego de la ejecución}). 

\begin{table}[H]
    \centering
    \begin{tabular}{l S[table-format=2.2] S[table-format=1.4] c}
        \toprule
        Distribución & {$\chi^{2}$} & {$p$-valor} & Decisión ($\alpha=0.05$) \\
        \midrule
        Uniforme & {} & {} & {}\\
        Exponencial & {} & {} & {}\\
        Gamma & {} & {} & --\\
        Normal & {} & {} & {}\\
        Binomial & {} & {} & {}\\
        Poisson & {} & {} & {}\\
        Empírica & {} & {} & {}\\
        \bottomrule
    \end{tabular}
    \caption{Resultados de las pruebas de bondad de ajuste. 
    Sustituir los blancos con los valores obtenidos.}
    \label{tab:chi2}
\end{table}

\section{Conclusiones}
\begin{itemize}
    \item Los algoritmos implementados generan muestras que, salvo
    pequeñas fluctuaciones, resultan consistentes con las distribuciones teóricas.
    \item El enfoque modular en Python simplifica la extensión a 
    nuevas distribuciones y el ajuste de parámetros.
    \item Las pruebas $\chi^{2}$ y los histogramas constituyen un mecanismo
    práctico para validar los generadores antes de utilizarlos en experimentos
    de simulación más complejos.
\end{itemize}

\section*{Agradecimientos}
Se agradece a la cátedra de Simulación (UTN--FRRO) por la guía
y al autor Thomas~Naylor por el material de referencia.

\appendix
\section{Código fuente completo}
\label{app:codigo}
\lstinputlisting[language=Python]{pruebas.py}

\begin{thebibliography}{9}
\bibitem{naylor1982}
Naylor, T.~H. \emph{Técnicas de simulación en computadoras}. El~Ateneo, 1982.
\end{thebibliography}

\end{document}
